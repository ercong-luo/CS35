% This LaTeX file contains your written lab questions.  You may answer these
% questions just by inserting your answer into this document.  You are not
% *required* to do your homework in LaTeX, but it's quite likely to be easier
% than e.g. the equation editor in OpenOffice Writer or Microsoft Word.
%
% If you're unfamiliar with LaTeX, see the document LearningLaTeX.tex in this
% same directory.  It contains a brief explanation and a few snippets of LaTeX
% code to get you started; in fact, it should have everything you need to
% complete this assignment.
\documentclass{article}

\usepackage{amsmath}
\usepackage{amssymb}
\usepackage{algpseudocode}
\usepackage{algorithmicx}
\usepackage{tikz}

\begin{document}

\section{AVL Trees}

\noindent \textbf{Problem 1.} Perform a right rotation on the root of the following tree.  Be sure to specify the subtrees used in the rotation.

\input{written-trees/problem1.1} % This is a lot like a #include in C++: it brings in the contents of problem1.1.tex and puts it here.

\noindent \textbf{Problem 2.} Show the left rotation of the subtree rooted at 27.  Be sure to specify the subtrees used in the rotation.

\input{written-trees/problem1.2}

\noindent \textbf{Problem 3.} Using the appropriate AVL tree algorithm, insert the value 47 into the following tree.  Show the tree before and after rebalancing.

\input{written-trees/problem1.3}

\noindent \textbf{Problem 4.} Using the appropriate AVL tree algorithm, remove the value 24 from the following tree.  Show the tree before and after \textit{each} rebalancing.

\input{written-trees/problem1.4}

\end{document}
